%----------------------------------------------------------------------------------------
%	PACKAGES AND OTHER DOCUMENT CONFIGURATIONS
%----------------------------------------------------------------------------------------

\documentclass[10pt]{article}
\usepackage{lipsum} % Package to generate dummy text throughout this template

%\usepackage[light, math]{iwona}
%\usepackage[sc]{mathpazo} % Use the Palatino font
\usepackage[T1]{fontenc} % Use 8-bit encoding that has 256 glyphs
\linespread{1.05} % Line spacing - Palatino needs more space between lines
\usepackage{microtype} % Slightly tweak font spacing for aesthetics
\usepackage{bm} % bold math

\usepackage[hmarginratio=1:1,top=32mm,columnsep=20pt]{geometry} % Document margins
\usepackage{multicol} % Used for the two-column layout of the document
\usepackage[hang, small,labelfont=bf,up,textfont=it,up]{caption} % Custom captions under/above floats in tables or figures
\usepackage{booktabs} % Horizontal rules in tables
\usepackage{float} % Required for tables and figures in the multi-column environment - they need to be placed in specific locations with the [H] (e.g. \begin{table}[H])

\usepackage{lettrine} % The lettrine is the first enlarged letter at the beginning of the text
\usepackage{paralist} % Used for the compactitem environment which makes bullet points with less space between them

\usepackage{abstract} % Allows abstract customization
\renewcommand{\abstractnamefont}{\normalfont\bfseries} % Set the "Abstract" text to bold
\renewcommand{\abstracttextfont}{\normalfont\small\itshape} % Set the abstract itself to small italic text

\usepackage{titlesec} % Allows customization of titles
\renewcommand\thesection{\Roman{section}} % Roman numerals for the sections
\renewcommand\thesubsection{\Roman{subsection}} % Roman numerals for subsections
\titleformat{\section}[block]{\large\scshape\centering}{\thesection.}{1em}{} % Change the look of the section titles
\titleformat{\subsection}[block]{\large}{\thesubsection.}{1em}{} % Change the look of the section titles

\usepackage{fancybox, fancyvrb, calc}
\usepackage[svgnames]{xcolor}


%----------------------------------------------------------------------------------------
%	DOCUMENT ID (Department, Professor, Course, etc.) 
%----------------------------------------------------------------------------------------

\usepackage{fancyhdr} % Headers and footers
\pagestyle{fancy} % All pages have headers and footers
\fancyhead{} % Blank out the default header
\fancyfoot{} % Blank out the default footer
\fancyhead[C]{Probabilidad y Likelihood} % Custom header text
\fancyfoot[RO,LE]{\thepage} % Custom footer text

%----------------------------------------------------------------------------------------
%	MY PACKAGES 
%----------------------------------------------------------------------------------------

\usepackage{amsmath}	
\usepackage[makeroom]{cancel}
%\usepackage{rotating}
\usepackage{textcomp}
\usepackage{caption}
\usepackage{etex}
%\usepackage[export]{adjustbox}
%\usepackage{afterpage}
%\usepackage{filecontents}
\usepackage{color}
\usepackage{latexsym}
\usepackage{lscape}				%\begin{landscape} and \end{landscape}
\usepackage{amsfonts}
%\usepackage{mathabx}
\usepackage{amssymb}
%\usepackage{dashrule}
%\usepackage{txfonts}
%\usepackage{pgfkeys}
%\usepackage{framed}
\usepackage{tree-dvips}
\usepackage{caption}
%\usepackage{fancyvrb}
%\usepackage{pgffor}
\usepackage{xcolor}
%\usepackage{pxfonts}
\usepackage{wasysym}
\usepackage{authblk}
%\usepackage{paracol}
\usepackage{setspace}
%\usepackage{qtree}
%\usepackage{tree-dvips}
\usepackage{sgame}				% shouldn't have neither array nor tabularx packages
\usepackage{tikz}
%\usetikzlibrary{trees}
\usepackage[latin1]{inputenc}
%\label{tab:1} 		%\autoref{tab:1}	%ocupar para citar.
% \hyperlik{table1}	\hypertarget{table1} 
% \textquoteright			%apostrofe
\usepackage{hyperref} 		%desactivar para link rojos





%----------------------------------------------------------------------------------------
%	Other ADDS-ON
%----------------------------------------------------------------------------------------

% independence symbol \independent
\newcommand\independent{\protect\mathpalette{\protect\independenT}{\perp}}
\def\independenT#1#2{\mathrel{\rlap{$#1#2$}\mkern2mu{#1#2}}}


% VERBATIM WITH BACKGROUND COLOR
\newenvironment{colframe}{%
  \begin{Sbox}
    \begin{minipage}
      {\columnwidth%-\leftmargin-\rightmargin-6pt
      }
    }{%
    \end{minipage}
  \end{Sbox}
  \begin{center}
    \colorbox{LightSteelBlue}{\TheSbox}
  \end{center}
}


\hypersetup{
    bookmarks=true,         % show bookmarks bar?
    unicode=false,          % non-Latin characters in Acrobat$'$s bookmarks
    pdftoolbar=true,        % show Acrobat$'$s toolbar?
    pdfmenubar=true,        % show Acrobat$'$s menu?
    pdffitwindow=false,     % window fit to page when opened
    pdfstartview={FitH},    % fits the width of the page to the window
    pdftitle={My title},    % title
    pdfauthor={Author},     % author
    pdfsubject={Subject},   % subject of the document
    pdfcreator={Creator},   % creator of the document
    pdfproducer={Producer}, % producer of the document
    pdfkeywords={keyword1} {key2} {key3}, % list of keywords
    pdfnewwindow=true,      % links in new window
    colorlinks=true,       % false: boxed links; true: colored links
    linkcolor=ForestGreen,          % color of internal links (change box color with linkbordercolor)
    citecolor=ForestGreen,        % color of links to bibliography
    filecolor=ForestGreen,      % color of file links
    urlcolor=ForestGreen           % color of external links
}


% PROPOSITIONS
\newtheorem{proposition}{Proposition}

%\linespread{1.5}

\usepackage[american]{babel}
\usepackage{csquotes}
\usepackage[backend=biber,style=authoryear,dashed=false,doi=false,isbn=false,url=false,arxiv=false]{biblatex}
%\DeclareLanguageMapping{american}{american-apa}
\addbibresource{/Users/hectorbahamonde/Bibliografia_PoliSci/library.bib} 
\addbibresource{/Users/hectorbahamonde/Bibliografia_PoliSci/Bahamonde_BibTex2013.bib} 


%----------------------------------------------------------------------------------------
%	TITLE SECTION
%----------------------------------------------------------------------------------------

%\title{\vspace{-15mm}\fontsize{18pt}{7pt}\selectfont\textbf{Experimental Economists and Psychologists: Two Worlds Apart}} % Article title

%\author[1]{
%\large
%\textsc{H\'ector Bahamonde}\\ 
%\thanks{}
%\normalsize Political Science Dpt. $\bullet$ Rutgers University \\ % Your institution
%\normalsize \texttt{e:}\href{mailto:hector.bahamonde@rutgers.edu}{\texttt{hector.bahamonde@rutgers.edu}}\\
%\normalsize \texttt{w:}\href{http://www.hectorbahamonde.com}{\texttt{www.hectorbahamonde.com}}
%\vspace{-5mm}
%}
%\date{\today}

%----------------------------------------------------------------------------------------

\begin{document}

%\maketitle % Insert title


\thispagestyle{fancy} % All pages have headers and footers

%----------------------------------------------------------------------------------------
%	ABSTRACT
%----------------------------------------------------------------------------------------

%\begin{abstract}
%	ABSTRACT
%\end{abstract}


%----------------------------------------------------------------------------------------
%	CONTENT
%----------------------------------------------------------------------------------------

%\graphicspath{
%{/Users/hectorbahamonde/RU/Term5/Experiments_Redlawsk/Experiment/Data/}
%}
\hspace{-5mm}{\bf Profesor}: H\'ector Bahamonde, PhD.\\
\texttt{e:}\href{mailto:hector.bahamonde@uoh.cl}{\texttt{hector.bahamonde@uoh.cl}}\\
\texttt{w:}\href{http://www.hectorbahamonde.com}{\texttt{www.hectorbahamonde.com}}\\
{\bf Curso}: MLE.\\
\hspace{-5mm}{\bf TA}: Gonzalo Barr\'ia.


\section*{Probabilidad y Likelihood}

Usualmente, nosotros estimamos la relaci\'on entre $x$ e $y$ v\'ia un modelo lineal OLS (\autoref{ols}).

\begin{equation}\label{ols}
y_{i} = \alpha + \beta_{1}x_{1} + \epsilon_{i}
\end{equation}

\paragraph{} En este setup, siempre hablamos de ``probabilidad''. Es claro que \autoref{ols} nos muestra que los valores de $y_{i}$ dependen de $\bm{x}$ (que es una matrix que contiene $\alpha$ y todos los $x$'s), m\'as el residuo $\epsilon$. Cada par\'ametro tiene un statement de probabilidad (recuerden el \emph{p-value}). 

\paragraph{} Es decir, la probabilidad de $y$ depende de $\bm{x}$, lo que significa que en este paradigma, ${\mathbf y}$ {\bf depende de un modelo} ($M$). Recuerda que cada vez que pones una $x$ a la mano derecha de la igualdad, est\'as haciendo un modelo (est\'as especificando que la realidad depende de ciertas variables independientes). M\'as formalmente, esta noci\'on se demuestra en \autoref{prob}:

\begin{equation} \label{prob}
Pr(y|M) = Pr(\text{datos}|\text{modelo})
\end{equation}

donde en \autoref{prob} $M$  es asumido ``fijo'' o ``conocido'' (``fixed'') y los datos ``random''. \emph{Lo son?} Recuerda que incluso existe la noci\'on del \emph{true model} (un modelo verdadero que existe ``por ah\'i'', pero que es inaccesible para nosotros).  \textcite[16]{King1998} explica que cuando ``$M$ is treated as given is a problem because uncertainty in inference lies with the model, not the data.'' 

\begin{itemize}
  \item {\bf $M$: no puede ser tratado como conocido}. Los modelos se construyen. Nadie viene a darnos el modelo correcto.
  \item {\bf datos}: no pueden ser \emph{random}. La naturaleza ya ha producido los datos. Es m\'as, y por esta misma raz\'on, los datos debieran ser considerados como ``dados'' o ``conocidos''. No? 
\end{itemize}

\paragraph{} Qu\'e diferencias existen entre los conceptos de \emph{probabilidad} y \emph{likelihood}?

\subsection*{Probabilidad: Incertidumbre Absoluta}

Las probabilidades, todas, ``viven'' entre el 0 y 1. 

\begin{equation} \label{prob:model}
\begin{split}
Y & \sim f(y|\theta, \alpha) \\
\bm{\theta} & = g(\bm{X}, \bm{\beta})
\end{split}
\end{equation}

donde la primera linea see lee ``$y$ se distribuye como una funci\'on de $y$ que depende de $\bm{\theta}$ (que es $\hat{y}$, o el valor esperado de $y_{i}$, y $\alpha$''. El valor $\alpha$ se puede entender como parametro que est\'a presente s\'olo en algunos modelos GLMs, o est\'a dado por otros factores. Por ejemplo, $\sigma^{2}$ (la varianza) est\'a dada, no se estima (pero se asume que es {\bf constante}). La segunda l\'inea es una implicaci\'on de la primera, y se lee asi: ``donde $\bm{\theta}$ es igual a una funcion $g$, donde metemos los datos observados $\bm{X}$ y calculamos los parametros $\bm{\beta}$''.

\paragraph{} Lo unico que estamos haciendo, es hacer la transici\'on a la notaci\'on de MLE. En este caso, estamos viendo d\'onde ``viven'' los par\'ametros y los datos observados y predichos, pero en lenguaje MLE. Si te fijas, $\bm{\theta}$ (la caja que contiene $\bm{X}$ y $\bm{\beta}$) es tomado como ``true parameter''. {\color{red}Si el modelo es tomado como ``verdadero'', qu\'e implicancias tiene el estimado $\beta$? Piensa en los intervalos de confianza}.

\subsection*{\emph{Likelihood}: Incertidumbre Relativa}

En lenguaje de likelihood (que se expresa con el car\'acter griego $\bm{\theta}$) se refiere a un par\'ametro $\theta_{i}$ de los infinitos par\'ametros $\bm{\theta}$ entre todos los modelos posibles ($\bm{\tilde{\theta}}$). Recuerda, en \autoref{prob:model} $\bm{\beta}$ se refiere a los ``true values''. Eso quiere decir que OLS y MLE tienen llegadas distintas: en MLE existen tantos $\bm{\theta}$ como hay datos. Esto implica que la incertidumbre en MLE est\'a anclada a los datos (no al modelo). En otras palabras, en likelihood la incertidumbre es \emph{relativa} porque el likelihood $\theta$ no depende del modelo ($\bm{\tilde{\theta}}$) sino que de los datos. Es por esto que en likelihood uno piensa en t\'erminos de qu\'e modelo ($\tilde{\theta}$) \emph{{\bf maximiza}} el likelihood de haber creado los datos $y$. {\bf A diferencia del lenguaje de probabilidad, los datos no son tomados como ``dados''.}


 


\begin{equation} \label{lik:model}
\begin{split}
L(\tilde{\theta}|y) & =  k(y)Pr(y|\tilde{\theta})\\
& \propto Pr(y|\tilde{\theta})
%\theta & = g(X, \beta)
\end{split}
\end{equation}

donde $k(y)$ es una funci\'on constante positiva desconocida que depende de $y$ (y cambia cada vez que $y$ cambia). Si te fijas, en probabilidad (\autoref{prob}) la probabilidad es $Pr(y|M)$, pero en likelihood (\autoref{lik:model}) la probabilidad de $Pr(y|\tilde{\theta})$ es ``proporcional'' (i.e. $\propto$) a $L(\tilde{\theta}|y)$. F\'ijate c\'omo se invierten los t\'erminos.

\paragraph{} Esto implica que los likelihoods de dos modelos pueden ser comparados, pero s\'olo si bienen de los mismos datos (recuerda que $k(y)$ depende de $y$). Al contrario, en probabilidad, la probabilidad de que algo pase, \emph{siempre} va entre 0 y 1 (independiente de los datos o los modelos usados). En cambio, los likelihoods pueden tomar cualquier n\'umero y no son necesariamente comparables (son como los errores est\'andard en un modelo lineal---s\'olo toman significancia dentro de ese modelo en particular).


%\newpage
\paragraph{}
\paragraph{}
\pagenumbering{Roman}
\setcounter{page}{1}
\printbibliography


\end{document}

